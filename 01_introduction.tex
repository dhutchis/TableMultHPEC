%!TEX root =SpGEMM_ACCUMULO_HPEC.tex

\section{Introduction}
\label{sIntro}
% 

Accumulo is well-known as a high performance NoSQL database with respect to ingest and scans.
The next step past ingesting and scanning is computing---running algorithms and analytics.
The advantage of doing computation in a database like Accumulo %, 
%as opposed to YARN or MapReduce directly on HDFS, 
is \emph{selective access}, \emph{data locality} and \emph{infrastructure reuse}.
Accumulo's features as a database enable fast access to data subsets and queries along indexed attributes.
Further, Accumulo sits atop the physical location data is stored and cached, such that computation inside
Accumulo tablet servers occurs local to data storage and avoids unnecessary network transfers,
effectively moving ``compute to data'' in contrast to client-server models that move ``data to compute.''
Computation within Accumulo also reuses its distributed infrastructure
(such as write-ahead logging, fault-tolerant execution atop Zookeeper and 
horizontal scalability from masters and tablet servers), 
which in environments already using Accumulo, are already deployed in production clusters,
saving the need to install and maintain a new distributed system for computation.

One form of computation commonly run on database data is linear algebra.
Researchers in the GraphBLAS Forum \cite{x} have identified a set of primitive operations 
that form the basis for linear algebraic algorithms useful for graphs, including 
Sparse General Matrix Multiplication (SpGEMM),  Sparse Element-wise Multiplication (SpEWiseX),
Sparse Reference of a subset (SpRef), Reduction along a dimension (Reduce),
Function application (Apply) and others.
Graphulo is an effort to realize the GraphBLAS primitives 
and enable algorithms in the language of linear algebra server-side on Accumulo \cite{x}.

%I don't want to make something like MapReduce-- Accumulo facilitates a particular kind of computation 
%using iterators.  Not all computation patterns fit into the iterator framework. EXAMPLE
%We shouldn't stuff computation contradictory to the iterator framework into Accumulo, 
%as many have done stretching iterative algorithms into the MapReduce framework.

In this paper we focus on implementing SpGEMM, a core operation at the heart of GraphBLAS.
%SpGEMM is arguably the most challenging GraphBLAS operation
In fact, most of the other GraphBLAS primitives can be expressed in terms of SpGEMM.
We can realize SpEWiseX by multiplying the transpose of the first matrix with the second
and using a custom multiplication function that only acts on elements whose row from the first 
matches the column of the second and returns multiplied elements with the using the row and column of the second.
\todo{Too much detail for intro}
We can realize SpRef when selecting a subset of rows by multiplying on the left with an identity matrix containing those rows,
and we can realize it when selecting a subset of column by multiplying similarly on the right.
We can realize Reduce by multiplying on the left or right depending on the dimension of interest by a dense vector of all ones.
%backbone of many linear algebraic algorithms.
One can even view graph search \cite{kepner2011graph} and table joins as applications of SpGEMM \cite{x}.

We call our implementation of SpGEMM on Accumulo TableMult, short for multiplication of Accumulo tables.
Accumulo tables have many similarities to sparse matrices, though a more exact analogy is to Associative Arrays 
\cite{kepner2014gabb}.


%Like most databases, Accumulo is often used as a powerful, distributed, indexed key-value data storage service.
%Programs read data from Accumulo and use it in further processing and/or write data to Accumulo that is 
%raw data or the result of processing. 
%Many users need to process data from Accumulo further before 
%This use of Accumulo works well as when all one needs is a key-value store, 
%but when the processing around data from Accumulo becomes complex, the user needs to create distributed computational
%infrastructure 

\todo[inline]{alternate wording below:}
Enrichment \cite{x} is the process of taking an input data set and transforming it into a new dataset, such as by 
applying a function or by fusing it with another dataset.
If Accumulo is restricted to traditional database use as a distributed, indexed key-value data storage service,
then a ``client'' must query Accumulo for necessary input data, apply the enrichment, and write the new data 
to Accumulo.  
Complex enrichment processes require complex client infrastructure to process large volumes of data efficiently,
resulting in the need for distributed computation engines such as Spark \cite{x}.
%Robust distributed computational infrastructure requires implementing service monitoring, 
Accumulo has many of these distributed features already implemented and, when used as a database, 
already deployed in the production clusters. 
Instead of duplicating Accumulo's infrastructure and well-optimized scale-out parallelism
in creating a new distributed computational system, we aim to shift computation 
to within Accumulo where appropriate,
as well as the possibility to perform computation more local to data storage.
Some would call this concept server-side computation, though in the case of Accumulo we might call it 
database-side computation.



\subsection{Paper Outline}



\section{Background}
\label{sBackground}







\subsection{Accumulo Iterators}
Accumulo includes a mechanism to perform limited serer-side computation called the 
\emph{iterator stack}.  Iterators inside the iterator stack are objects of classes
that implement the \texttt{SortedKeyValueIterator} (SKVI) interface, an interface
reminiscent of but more complex than built-in Java iterators from the \texttt{java.lang} package
in that they have methods to return a current entry (\texttt{getTopKey} and \texttt{getTopValue})
and proceed to the next entry (\texttt{next}) until no more entries remain (\texttt{hasTop}).
Iterators may hold state initialized in \texttt{init}, to which Accumulo hands 
options of the form \texttt{Map<String,String>} passed from the client.%, though the state should 
%be arranged in such a way as to

Arguably, the most critical component of an iterator is its \texttt{seek} method,
which instructs an iterator to jump to the beginning of a passed-in range. System iterators 
at the top of an iterator stack perform actual disk seeks and cache locations in memory when seeked.

During a scan, Accumulo constructs an iterator stack for each tablet whose keys overlap some portion 
of the scan range. These iterator stacks may run in parallel, and each is seeked to the range of 
keys in the current tablet, instersected with the scan range. When any call to the iterator stack 
returns, Accumulo may choose to destroy the iterator stack and later re-create it,
passing a new seek range starting at the last key returned from \texttt{getTopKey}, exclusive.
Accumulo does this when it needs to switch data sources (such as RFiles) after a compaction, 
when a client stops requesting data, or out of fairness to other concurrent scans.

Iterators do not have full lifecycle control in that there is no \texttt{close} method 
that allows an iterator to clean up its state before being destroyted. The only safe way for an 
iterator to use state requiring cleanup, such as opening a file or starting a thread,
is for the iterator to clean up its state before returning from a method call.
Ideas discussed in \cite{ACCUMULO-3751} may lax this restriction for future Accumulo versions.

%\1 Iterators are most commonly used for streaming computation in the sense that iterators
%should ideally run in a single pass over their source's data and not store too much state.
%\1 No lifecycle control.

