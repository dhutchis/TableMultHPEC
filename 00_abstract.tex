%!TEX root = Graphulo_MatrixMultiply_HPEC2015.tex
\begin{abstract}
The Apache Accumulo database excels at distributed storage
and indexing
and is ideally suited for storing graph data.
Many big data graph analytics
compute on graph data and persist the results back to the database.
These graph calculations are often best performed inside the database server. 
%The Graphulo library is designed to take advantage of 
%Accumulo's graph capabilities
%Users must subsequently implement complex clients and shuffle 
%data between Accumulo storage and computing engines.
%% Server-side computation is difficult in Accumulo 
%% due to its design for distributed storage and not for general computing,
%% yet many big data applications such as enrichment and analytics \todo{use more specific example}
%% compute on Accumulo data and persist results back to Accumulo anyway.
%% Users must subsequently implement complex clients and shuffle 
%% data between Accumulo storage and engines for compute.
The GraphBLAS standard provides a compact and efficient basis for
a wide range of graph applications through a small number of sparse matrix operations. 
%we propose 
In this article, We implement GraphBLAS sparse matrix multiplication
%inside the Accumulo database
server-side
by leveraging Accumulo's native, high-performance iterators.
We compare the mathematics and performance 
of inner and outer product implementations,
and show how an outer product implementation achieves optimal performance near
Accumulo's peak write rate.
% part of the Graphulo library
We offer our work as a core component to the Graphulo library
that will deliver matrix math primitives for graph analytics within Accumulo.
\end{abstract}
% IEEEtran.cls defaults to using nonbold math in the Abstract.
% This preserves the distinction between vectors and scalars. However,
% if the conference you are submitting to favors bold math in the abstract,
% then you can use LaTeX's standard command \boldmath at the very start
% of the abstract to achieve this. Many IEEE journals/conferences frown on
% math in the abstract anyway. 

