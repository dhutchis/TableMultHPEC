%!TEX root = SpGEMM_Accumulo_HPEC.tex
\begin{abstract}
Databases such as Apache Accumulo excel at distributed storage
and indexing
and are ideally suited for graphs.
%yet are not designed for general computing.
%However,
Many big data applications such as enrichment and analytics
compute on graph data and persist results back to the database.
%The Graphulo library is designed to take advantage of 
%Accumulo's graph capabilities
%Users must subsequently implement complex clients and shuffle 
%data between Accumulo storage and computing engines.
%% Server-side computation is difficult in Accumulo 
%% due to its design for distributed storage and not for general computing,
%% yet many big data applications such as enrichment and analytics \todo{use more specific example}
%% compute on Accumulo data and persist results back to Accumulo anyway.
%% Users must subsequently implement complex clients and shuffle 
%% data between Accumulo storage and engines for compute.
In this article, 
%we propose 
we enable sparse matrix multiplication
server-side on Accumulo tables
by repurposing Accumulo iterators.
We compare mathematics and performance 
of inner and outer product approaches 
and show how an outer product implementation scales with synthetic experiments.
% part of the Graphulo library
We offer our work as a core component to a Graphulo library
that will deliver linear algebra primitives within Accumulo.
\end{abstract}
% IEEEtran.cls defaults to using nonbold math in the Abstract.
% This preserves the distinction between vectors and scalars. However,
% if the conference you are submitting to favors bold math in the abstract,
% then you can use LaTeX's standard command \boldmath at the very start
% of the abstract to achieve this. Many IEEE journals/conferences frown on
% math in the abstract anyway. 

