\documentclass[conference]{IEEEtran}
\pdfoutput=1
\usepackage{cite}
\usepackage{url}
\usepackage{tikz}
\usepackage{footnote}
\usepackage{balance}
%\usepackage{titlesec}

%\ifCLASSINFOpdf
%  \usepackage[pdftex]{graphicx}
%\else
%  % or other class option (dvipsone, dvipdf, if not using dvips). graphicx
%  % will default to the driver specified in the system graphics.cfg if no
%  % driver is specified.
%  \usepackage[dvips]{graphicx}
%\fi
%declare the path(s) where your graphic files are
\graphicspath{{./img/}}
%and their extensions so you won't have to specify these with
% every instance of \includegraphics
\usepackage{epstopdf}
\DeclareGraphicsExtensions{.eps,.pdf,.png}
\usepackage[cmex10]{amsmath}
\hyphenation{op-tical net-works semi-conduc-tor linear}

% todo notes:
\usepackage{todonotes}

% to fix the figure* environment:
\usepackage{dblfloatfix} %provides: \usepackage{fixltx2e}

% for multi-figures:
\usepackage{threeparttable}
\usepackage{graphicx}
\usepackage{caption}
\usepackage{subcaption}
\usepackage[]{algorithm2e}
%\usepackage{amsmath}
\usepackage{amsfonts} % for \mathbb{R}
\usepackage{amssymb} % for \intercal
%\usepackage[margin=1.15in]{geometry}
%\usepackage{units} %\nicefrac
\usepackage{authblk}

\newcommand{\matr}[1]{\mathbf{#1}} % undergraduate algebra version
%\newcommand{\matr}[1]{#1}          % pure math version
%\usepackage{bm}
%\newcommand{\matr}[1]{\bm{#1}}     % ISO complying version
%\DeclareMathOperator{\fore}{foreach}
%\DeclareMathOperator{\emit}{emit}
\newcommand{\tr}[0]{{\intercal}} %\!
\newcommand{\col}[0]{\colon\!}

\usepackage{mathtools}
\DeclarePairedDelimiter\ceil{\lceil}{\rceil}
\DeclarePairedDelimiter\floor{\lfloor}{\rfloor}
\DeclarePairedDelimiter\paren{\left(}{\right)}

\usepackage{siunitx}
\sisetup{round-precision=2,round-mode=places,scientific-notation=true}

\usepackage{multirow}
\usepackage{adjustbox}
\usepackage{array}
\newcolumntype{R}[2]{%
    >{\adjustbox{angle=#1,lap=\width-(#2)}\bgroup}%
    l%
    <{\egroup}%
}
%\newcommand*\rot{\multirow{2}{R{45}{2em}}}% no optional argument here, please!


% to allow [H] for algorithms
% https://tex.stackexchange.com/questions/82271/multiple-algorithm2e-algorithms-in-two-column-documents/82272#82272
\makeatletter
\newcommand{\removelatexerror}{\let\@latex@error\@gobble}
\makeatother

% vspace above and below inline algorithms
\newlength{\algspace}
\setlength{\algspace}{3pt}

% MACROS:
%\newcommand{\numparties}{\ensuremath{\mathcal{P}}}

\newcommand{\matlab}{\textsc{Matlab}}


\makeatletter
\renewenvironment{thebibliography}[1]{%
%\vspace*{-6pt}
  \@xp\section\@xp*\@xp{\refname}%
  \normalfont\footnotesize\labelsep .5em\relax
  \renewcommand\theenumiv{\arabic{enumiv}}\let\p@enumiv\@empty
%\vspace*{-2pt}% NEW
  \list{\@biblabel{\theenumiv}}{\settowidth\labelwidth{\@biblabel{#1}}%
    \leftmargin\labelwidth \advance\leftmargin\labelsep
    \usecounter{enumiv}}%
  \sloppy \clubpenalty\@M \widowpenalty\clubpenalty
  \sfcode`\.=\@m
}{%
  \def\@noitemerr{\@latex@warning{Empty `thebibliography' environment}}%
  \endlist
}
\makeatother


\begin{document}

\title{Graphulo Implementation of Server-Side \\ Sparse Matrix Multiply in the Accumulo Database}


%% \author{
%% \IEEEauthorblockN{Dylan Hutchison, Vijay Gadepally, Jeremy Kepner, Adam Fuchs }
%% \IEEEauthorblockA{MIT Computer Science and Artificial Intelligence Laboratory\\
%% Cambridge, MA 02420\\
%% \{vijayg, kepner, bamiller\}@ll.mit.edu}}

\author[D. Hutchison et al.]
       {Dylan Hutchison,$^{{\dagger}*}\;$ Jeremy Kepner,$^{{\dagger}{\ddagger}{\diamond}*}\;$ Vijay Gadepally,$^{{\dagger}{\ddagger}*}\;$ Adam Fuchs$^+$ \\
         \\
         $^{\dagger}$MIT Lincoln Laboratory, 
         $^{\ddagger}$MIT Computer Science \& AI Laboratory, \\
         $^{\diamond}$MIT Mathematics Department, 
         $^+$Sqrrl, Inc. %\vspace{-1em}
       }

%% \author[*,$\dagger$]{Dylan Hutchison\thanks{dhutchis@mit.edu}}
%% \author[*]{Jeremy Kepner\thanks{kepner@csail.mit.edu}}
%% \author[*]{Vijay Gadepally\thanks{vijayg@csail.mit.edu}}
%% \author[$\dagger$]{Adam Fuchs\thanks{adam@sqrrl.com}}
%% \affil[*]{MIT Lincoln Laboratory}
%% \affil
%% \affil[$\dagger$]{Sqrrl}
%% \renewcommand\Authands{, }

% for over three affiliations, or if they all won't fit within the width
% of the page, use this alternative format:
%
%\author{\IEEEauthorblockN{Michael Shell\IEEEauthorrefmark{1},
%Homer Simpson\IEEEauthorrefmark{2},
%James Kirk\IEEEauthorrefmark{3},
%Montgomery Scott\IEEEauthorrefmark{3} and
%Eldon Tyrell\IEEEauthorrefmark{4}}
%\IEEEauthorblockA{\IEEEauthorrefmark{1}School of Electrical and Computer Engineering\\
%Georgia Institute of Technology,
%Atlanta, Georgia 30332--0250\\ Email: see http://www.michaelshell.org/contact.html}
%\IEEEauthorblockA{\IEEEauthorrefmark{2}Twentieth Century Fox, Springfield, USA\\
%Email: homer@thesimpsons.com}
%\IEEEauthorblockA{\IEEEauthorrefmark{3}Starfleet Academy, San Francisco, California 96678-2391\\
%Telephone: (800) 555--1212, Fax: (888) 555--1212}
%\IEEEauthorblockA{\IEEEauthorrefmark{4}Tyrell Inc., 123 Replicant Street, Los Angeles, California 90210--4321}}

\maketitle

{\let\thefootnote\relax\footnote{\hspace{-\parindent}Dylan Hutchison is the corresponding
    author, reachable at dhutchis@mit.edu.
}}
{\let\thefootnote\relax\footnote{*This material is based upon work
    supported by the National Science Foundation under Grant
    No. DMS-1312831. Opinions, findings, and conclusions or recommendations expressed in this material are those of the author(s) and do not necessarily reflect the views of the National Science Foundation.
}}
\setcounter{footnote}{0}
%!TEX root = SpGEMM_Accumulo_HPEC.tex
\begin{abstract}
Databases such as Apache Accumulo excel at distributed storage
and indexing
and are ideally suited for graphs.
%yet are not designed for general computing.
%However,
Many big data applications such as enrichment and analytics
compute on graph data and persist results back to the database.
%The Graphulo library is designed to take advantage of 
%Accumulo's graph capabilities
%Users must subsequently implement complex clients and shuffle 
%data between Accumulo storage and computing engines.
%% Server-side computation is difficult in Accumulo 
%% due to its design for distributed storage and not for general computing,
%% yet many big data applications such as enrichment and analytics \todo{use more specific example}
%% compute on Accumulo data and persist results back to Accumulo anyway.
%% Users must subsequently implement complex clients and shuffle 
%% data between Accumulo storage and engines for compute.
In this article, 
%we propose 
we enable sparse matrix multiplication
server-side on Accumulo tables
by repurposing Accumulo iterators.
We compare mathematics and performance 
of inner and outer product approaches 
and show how an outer product implementation scales with synthetic experiments.
% part of the Graphulo library
We offer our work as a core component to a Graphulo library
that will deliver linear algebra primitives within Accumulo.
\end{abstract}
% IEEEtran.cls defaults to using nonbold math in the Abstract.
% This preserves the distinction between vectors and scalars. However,
% if the conference you are submitting to favors bold math in the abstract,
% then you can use LaTeX's standard command \boldmath at the very start
% of the abstract to achieve this. Many IEEE journals/conferences frown on
% math in the abstract anyway. 



% For peerreview papers, this IEEEtran command inserts a page break and
% creates the second title. It will be ignored for other modes.
\IEEEpeerreviewmaketitle

% Reduce space for times; go back to original spacing after table
\let\stimes\times
\renewcommand{\times}[0]{{\,\stimes{}\,}}


%!TEX root =SpGEMM_ACCUMULO_HPEC.tex

\section{Introduction}
\label{sIntro}
% 

Accumulo is well-known as a high performance NoSQL database with respect to ingest and scans \cite{sen2013benchmarking}.
The next step past ingesting and scanning is computing---running enrichment, algorithms and analytics.
Advantages of doing computation in a database like Accumulo %, 
%as opposed to YARN or MapReduce directly on HDFS, 
are \emph{selective access}, \emph{data locality} and \emph{infrastructure reuse}.
Accumulo's features as a database enable fast access to data subsets and queries along indexed attributes.
Further, Accumulo sits atop the physical location data is stored and cached, such that computation inside
Accumulo tablet servers occurs local to data storage and avoids unnecessary network transfers,
effectively moving ``compute to data'' in contrast to client-server models that move ``data to compute.''
Computation within Accumulo also reuses its distributed infrastructure, 
such as write-ahead logging, fault-tolerant execution atop Zookeeper and 
horizontal scalability from masters load balancing tablets.
Why not reuse these services already deployed in production clusters that use Accumulo,
rather than install and maintain new distributed systems for computation?

%cite linear algebra on database data?
One family of computation commonly run on database data is linear algebra.
Researchers in the GraphBLAS Forum \cite{x} have identified a set of primitive operations 
that form a basis for linear algebraic algorithms useful for graphs, including 
Sparse General Matrix Multiplication (SpGEMM),  Sparse Element-wise Multiplication (SpEWiseX),
Sparse Reference of a subset (SpRef), Reduction along a dimension (Reduce),
Function application (Apply) and others.
Graphulo is an effort to realize the GraphBLAS primitives 
and enable algorithms in the language of linear algebra server-side on Accumulo \cite{x}.

%I don't want to make something like MapReduce-- Accumulo facilitates a particular kind of computation 
%using iterators.  Not all computation patterns fit into the iterator framework. EXAMPLE
%We shouldn't stuff computation contradictory to the iterator framework into Accumulo, 
%as many have done stretching iterative algorithms into the MapReduce framework.

In this paper we focus on implementing SpGEMM, a core operation at the heart of GraphBLAS.
%SpGEMM is arguably the most challenging GraphBLAS operation
In fact, most GraphBLAS primitives can be expressed in terms of SpGEMM.
We can realize SpEWiseX by multiplying the transpose of the first matrix with the second
and using a custom multiplication function that only acts on elements whose row from the first 
matches the column of the second and returns multiplied elements using the row and column of the second;
\todo{Too much detail for intro}
we can realize SpRef when selecting a row or column subset by multiplying on the left or right with an identity matrix 
containing those rows or columns;
we can realize Reduce by multiplying on the left or right %depending on the dimension of interest 
by a dense vector of all ones.
%backbone of many linear algebraic algorithms.
One can even view graph search \cite{kepner2011graph} and table joins \cite{x} as applications of SpGEMM.

We call our implementation of SpGEMM on Accumulo \textsc{TableMult}, short for multiplication of Accumulo tables.
Accumulo tables have many similarities to sparse matrices, though a more exact analogy is to Associative Arrays 
\cite{kepner2014gabb}.

We are particularly interested in SpGEMM for queued analytics, that is, analytics on a selected table subset.  
Queued analytics allows us to maximally take advantage of Accumulo as a database 
by quickly accessing a subset of interest, 
whereas whole-table analytics usually perform better on parallel file systems such as Lustre or Hadoop.
We therefore prioritize low latency over high throughput, %when we must choose between them
in the best case enabling analysts to manipulate Accumulo data interactively.
% explain why performance is whole-table

%\subsection{Paper Outline}

We review iterator stacks, Accumulo's model of server-side computation, in Section 2.
Section 3 formally defines matrix multiplication and compares inner and outer product formulations,
ultimately settling on outer product for our TableMult implementation.
We show TableMult's design as Accumulo iterators in Section 4,
and we test its scalability with experiments in Section 5.

%Background and Algorithms
%Primer


%\section{Background}
%\label{sBackground}







\section{Accumulo Server-side Computation: Iterators}
Accumulo includes a mechanism to perform limited serer-side computation called the 
\emph{iterator stack}.  Iterators inside the iterator stack are objects of classes
that implement the \texttt{SortedKeyValueIterator} (SKVI) interface, an interface
reminiscent of but more complex than built-in Java iterators %from the \texttt{java.lang} package
in that they have methods to return a current entry (\texttt{getTopKey} and \texttt{getTopValue})
and proceed to the next entry (\texttt{next}) until no more entries remain (\texttt{hasTop}).
Iterators may hold state initialized in \texttt{init}, to which Accumulo hands 
options of the form \texttt{Map<String,String>} passed from the client.%, though the state should 
%be arranged in such a way as to

Arguably, the most critical component of an iterator is its \texttt{seek} method,
which instructs an iterator to jump to the beginning of a passed-in range. System iterators 
at the top of an iterator stack perform actual disk seeks.% and cache locations in memory when seeked.

During a scan, Accumulo constructs an iterator stack for each tablet whose keys overlap some portion 
of the scan range. These iterator stacks may run in parallel, and each is seeked to the range of 
keys in the current tablet, instersected with the scan range. When any call to the iterator stack 
returns, Accumulo may choose to destroy the iterator stack and later re-create it,
passing a new seek range starting at the last key returned from \texttt{getTopKey}, exclusive.
Accumulo does this when it needs to switch data sources (such as RFiles) after a compaction, 
when a client stops requesting data, or out of fairness to other concurrent scans.

Iterators do not have full lifecycle control in that there is no \texttt{close} method 
that allows an iterator to clean up its state before being destroyted. The only safe way for an 
iterator to use state requiring cleanup, such as opening a file or starting a thread,
is for the iterator to clean up its state before returning from a method call.
Ideas discussed in \cite{ACCUMULO-3751} may lax this restriction for future Accumulo versions.

%\1 Iterators are most commonly used for streaming computation in the sense that iterators
%should ideally run in a single pass over their source's data and not store too much state.
%\1 No lifecycle control.


%!TEX root=SpGEMM_ACCUMULO_HPEC.tex

\section{TableMult Design}
\label{sDesign}


\subsection{Matrix Multiplication}
\label{sMatMul}
Given input matrices $\matr{A}$ of size $n \times m$, $\matr{B}$ of size $m \times p$,
and operations $\oplus$ and $\otimes$ representing summation and multiplication,
the matrix multiplication $\matr{A} \,{\oplus}.{\otimes}\, \matr{B} = \matr{C}$, or more shortly $\matr{A}\matr{B} = \matr{C}$,
defines entries of result matrix C as 
\[ \matr{C}(i,j) = \bigoplus_{k=1}^m \matr{A}(i,k) \otimes \matr{B}(k,j) \]
We call intermediary results of $\otimes$ operations \emph{partial products}.

In the case of sparse matrices, we only perform $\oplus$ and $\otimes$ operations where both operands are nonzero,
an optimization stemming from the requirement that 0 is an additive identity of $\oplus$ such that $a \oplus 0 = 0 \oplus a = a$,
and that 0 is a multiplicative annhilator of $\otimes$ such that $a \otimes 0 = 0 \otimes a = 0$.
Sparse arithmetic is impossible without these conditions, since in that case zero elements could generate nonzero results.

% In Accumulo, zero elements are entries that do not exist in a table. Entries that actually contain the value zero may exist
% from an operation producing a zero value (such as summing partial products 5, -3 and -2).  
% We currently deliver no special treatment to these entries and plan on adding
% an optional feature that removes them when encountered.


We study two well known patterns for matrix multiplication,
inner and outer product, in terms of how they implement $\otimes$,
deferring $\oplus$ to run on output generated from applying $\otimes$.
We use Matlab notation in pseudocode for arrays and indexing.

The more common inner product method runs the following:
\begin{algorithm}[h]
\SetKwBlock{fore}{for}{} 
\SetKw{emit}{emit}
\fore($i = 1\col n$){
\fore($j = 1\col p$){
\emit{$\matr{A}(i,:) \cdot \matr{B}(:,j)$}
}}
\end{algorithm}

\noindent where the operation $\cdot$ is inner (also called dot) product on vectors, which we may unfold as
\begin{algorithm}[h]
\SetKwBlock{fore}{for}{} 
\SetKw{emit}{emit}
\fore($i = 1\col n$){
\fore($j = 1\col p$){
\fore($k = 1\col m$){
\emit{$\matr{A}(i,k) \otimes \matr{B}(k,j)$}
}}}
\end{algorithm}

Inner product has an advantage of generating output ``in order,'' meaning that all partial products needed 
to compute a particular element $\matr{C}(i,j)$ are generated consecutively by the third-level loop.
We may apply the $\oplus$ operation immediately after each third-level loop and obtain an element in the result matrix.
This means that inner product is easy to ``pre-sum,'' an Accumulo term for applying a Combiner
locally before sending entries to a remote but globally-aware table combiner.
% move to another section?
%Inner product also generates output ``in order'' in a second sense: 
It is also adventageous that inner product generates entries sorted by row 
and column, which allows inner product to be used in standard iterator stacks that require sorted output.

Despite its order-preserving advantages, we chose not to implement inner product 
because it requires multiple passes:
the second-level loop that scans over all of input matrix $\matr{B}$
repeats for each row of $\matr{A}$ from the top-level loop iteration.
%The same effect holds if we flipped the process symmetrically.
Under our assumption that we cannot fit $\matr{B}$ entirely in memory,
multiple passes over $\matr{B}$ translates to multiple Accumulo scans that each require a disk read.
We found in our performance tests that multiple scans over $\matr{B}$ 
delivered over an order of magnitude worse performance, taking over 100 seconds to multiply SCALE 11 inputs
whereas the outer product method ran in under 8 seconds.

Outer product matrix multiply runs the following:
\begin{algorithm}
\SetKwBlock{fore}{for}{} 
\SetKw{emit}{emit}
\fore($k = 1\col m$){
\emit{$\matr{A}(:,k) \times \matr{B}(k,:)$}
}
\end{algorithm}

\noindent where the operation $\times$ is outer (also called tensor or Carteisan) product on vectors, which we may unfold as
\begin{algorithm}
\SetKwBlock{fore}{for}{} 
\SetKw{emit}{emit}
\fore($k = 1\col m$){
\fore($i = 1\col n$){
\fore($j = 1\col p$){
\emit{$ \matr{A}(i,k) \otimes \matr{B}(k,j)$}
}}}
\end{algorithm}

Outer product emits partial products in unsorted order.
This is due to moving the $i$ and $j$ loops
that determine partial product position
below the top-level $k$ loop.

On the other hand, outer product only requires a single pass over both input matrices.
This is because the top-level $k$ loop fixes a dimension of both $\matr{A}$ and $\matr{B}$.
Once we finish processing a full column of $\matr{A}$ and row of $\matr{B}$,
we never need to read them again (i.e., we never need to restart the top-level $k$ loop).

In terms of memory usage, outer product works best when either the matching row or column fits in memory.
If neither fits, then we could still run the algorithm by
re-reading rows of $\matr{B}$ while streaming through columns of $\matr{A}$ 
(or vice versa by symmetry of $i$ and $k$).
%as suggested by the second-level loop repeating the third-level loop running through rows of $\matr{B}$.
%Storing one row in memory is a much lower cost than storing a whole matrix in memory.
We may implement the ``no memory assumption'' streaming approach in Accumulo by using
\texttt{deepCopy} SKVI methods to store ``pointers'' to the beginning of rows from table $\matr{B}$
(or columns of table $\matr{A}$),
perhaps at the cost of extra disk reads.
%However, this strategy may come at the cost of extra disk seeks, and so
%we leave testing its performance to future work, for now storing the current row of table $\matr{B}$ in memory.

Because $k$ runs along the second dimension of $\matr{A}$
and Accumulo uses a row-oriented data layout, we implement 
TableMult to operate on $\matr{A}$'s transpose $\matr{A^\tr}$.



\subsection{TableMult Iterators}
\label{sTableMult}

We implement SpGEMM with three iterators placed on a BatchScan of table $\matr{B}$:
RemoteSourceIterator, TwoTableIterator and RemoteWriteIterator.
The BatchScanner directs Accumulo to run the iterators on tablets of $\matr{B}$ in parallel.
%The: We place these iterators on  scan itself emits no entries except for a smidgeon of ``monitoring entries'' 
%that inform the client about TableMult progress. 
%% Instead, the scan on table $\matr{B}$ reads from table $\matr{A}$T
%% by opening a Scanner and writes to result table $\matr{C}$
%% by opening a BatchWriter, all within the scan's iterator stack.
%% See Figure~\ref{fIteratorStackSpGEMM} for an illustration.

The key idea behind the TableMult iterators is that they divert normal dataflow by opening a BatchWriter,
%reducing entries sent to the client to zero and instead sending 
redirecting entries out-of-band to a result table via %result table $\matr{C}$
Accumulo's standard ingest channel that does not require sorted order. 
The scan itself emits no entries except for a smidgeon of ``monitoring entries'' 
that inform the client about TableMult progress.
%Thus the TableMult iterators act as reduction iterators, even though they actually transmit 
%a huge stream of entries out-of-band to another Accumulo table.
We enable multi-table iterator dataflow by opening Scanners 
that read remote Accumulo tables out-of-band.
%which in the case of SpGEMM means scanning table $\matr{A^\tr}$.
Scanners and BatchWriters are standard tools for Accumulo clients; 
by creating them inside Accumulo iterators, we enable client-side processing patterns
within the Accumulo server.

We illustrate TableMult's data flow in Figure~\ref{fIteratorStackSpGEMM},
placing a Scanner on table $\matr{A^\tr}$
and a BatchWriter on result table $\matr{C}$.

\begin{figure}[htb]
\centering
\includegraphics[width=3.2in]{TableMultIteratorStack}
\caption{Data flow through the TableMult iterator stack}
\label{fIteratorStackSpGEMM}
\end{figure}

\subsubsection{RemoteSourceIterator}
RemoteSourceIterator scans an Accumulo table
(not necessarily in the same cluster) with the same range it is seeked to.
Clients pass connection information including zookeeper hosts, timeout,
username and password via iterator options. %in the form of a \texttt{Map<String,String>}.
We leave more secure scan authentication methods to future work,
although only users with access to the Accumulo instance may see the password in iterator options.

We also use iterator options to specify row and column subsets, 
encoding them in a string format similar to that in D4M \cite{kepner2012dynamic}.
Row subsets are straightforward since Accumulo uses row-oriented indexing.
Column subsets can be implemented with filter iterators
but do not improve performance since Accumulo must read every column from disk.
We encourage users to maintain a tranpose table
using strategies similiar to the D4M Schema \cite{kepner2013d4m}
for cases requiring column indexing.

Multiplying table subsets is crucial for queued analytics on selected rows.
However for simpler performance evaluation, 
our experiments in Section~\ref{sPerformance} multiply whole tables.

\subsubsection{TwoTableIterator}
TwoTableIterator reads from two iterator sources, one for $\matr{A^\tr}$ and one for $\matr{B}$,
and performs the core operations of the outer product algorithm in three phases:
\begin{enumerate}
\item Align Rows.  Read entries from $\matr{A^\tr}$ and $\matr{B}$ until they advance to a matching row
or one runs out of entries. We skip non-matching rows 
since they would multiply with an all-zero row that, by Section~\ref{sMatMul}'s assumptions,
generate all zero patrial products.
\item Cartesian product. Read both matching rows into an in-memory data structure. 
Initialize an iterator that emits pairs of entries from the rows' Carteisan product.
\item Multiply. Pass pairs of entries to $\otimes$ and emit results. 
\end{enumerate}

The user defines $\otimes$ by specifying a class that implements a provided ``multiply interface''
which TwoTableIterator instantiates and calls. 
For our experiments we implement $\otimes$ as \texttt{java.math.BigDecimal} multiplication
which guarantees correctness under large or precise real numbers.
BigDecimal decoding did not noticeably impact performance.

\subsubsection{RemoteWriteIterator}
RemoteWriteIterator writes entries to a remote Accumulo table using a BatchWriter. %created on \texttt{init}.
Entries do not have to be in sorted order because Accumulo sorts incoming entries as part of its
standard ingest process. Like RemoteSourceIterator, the client passes connection information 
for the remote table via iterator options.

Barring extreme events such as exceptions in the iterator stack or thread death,
we designed RemoteWriteIterator to maintain correctness, such that entries generated from
RemoteWriteIterator's source write to the remote table once.
We accomplish this by performing all BatchWriter operations within a single function call
before ceding thread contol back to the Accumulo tablet server.  

A performance concern remains in the case of TableMult over a subset of the input tables' rows 
that consists of many disjoint ranges, such as 1M ``singleton'' ranges spanning one row each.
It is inefficient to flush the BatchWriter before returning from each seek call, which happens once per 
disjoint scan range, and a known Accumulo issue could even crash the tablet server \cite{ACCUMULO-3710}.
We accomodate this case by ``transfering seek control'' over the desired row range
subset from the Accumulo tablet server to RemoteWriteIterator by passing the range objects through 
iterator options using the same techniques as RemoteSourceIterator, as opposed to the usual method of 
passing range objects to the %\texttt{setRanges} 
BatchScanner on table $\matr{B}$.
RemoteWriteIterator then traverses all ranges in the desired subset 
(within the tablet it runs on) within one call to seek.
%In the case of multiple tablets for table $\matr{B}$, RemoteWriteIterator running on each tablet handles 
%the portion of ranges that intersects with the range of keys in its tablet.

We include an option to BatchWrite the result table's transpose in place of or alongside
the result table. Writing the transpose facilitates chaining TableMults one after another
as well as maintenance of transpose indexing.

\subsubsection{Lazy $\oplus$}
We lazily sum partial products by placing a Combiner subclass implementing BigDecimal addition 
on table $\matr{C}$ at scan, minor and major compaction scopes.
Thus, $\oplus$ occurs sometime after RemoteWriteIterator writes partial products to $\matr{C}$
yet necessarily before entries from $\matr{C}$ may be seen such that we always achieve correctness.
Summation could happen when Accumulo flushes table $\matr{C}$'s entries cached in memory to a new RFile, 
when Accumulo compacts RFiles together or when a client scans $\matr{C}$. 

The key algebraic requirement for implementing $\oplus$ inside a Combiner
is that $\oplus$ must be associative and commutative.
These properties allow us to perform $\oplus$ on subsets of a result element's partial products 
and on any ordering of them, which is chaotic by the outer product's nature.
If we truly had an $\oplus$ operation that required seeing all partial products at once,
we would have to either gather partial products at the client or initiate a full major compaction.

\subsubsection{Monitoring}
RemoteWriteIterator never emits entries to the client by default. 
One downside of this approach is that clients cannot precisely track the progress of a TableMult operation.
The only information clients would have are scan and write rates from the Accumulo monitor,
whether a scan is running, idle or queued from the tablet server, and what partial products 
are written so far from scanning the result table.

We therefore implemented a monitoring option that emits a value
containing the number of entries TwoTableIterator processed
at a client-set frequency.
RemoteWriteIterator emits monitoring entries at ``safe'' points, that is,
points at which we can recover the iterator stack's state 
if Accumulo destroys, re-creates and re-seeks it.
%to a range starting from its last emitted key.
Stopping after emitting the last value in the outer product of two rows is safe 
because we place the last value's row key in the monitoring key and know, 
in the event of an iterator stack rebuild, to proceed to the next matching row.
We are also experimenting with stopping in the middle of an outer product by encoding the 
column family and qualifier of input tables' keys in the monitoring key.


%%\subsubsection{Client API}
%% The following shows how client programs call TableMult in Java:

%% %\todo[inline]{Put Java signature of TableMult call?}

%% \definecolor{javared}{rgb}{0.6,0,0} % for strings
%% \definecolor{javagreen}{rgb}{0.25,0.5,0.35} % comments
%% \definecolor{javapurple}{rgb}{0.5,0,0.35} % keywords
%% \definecolor{javadocblue}{rgb}{0.25,0.35,0.75} % javadoc
 
%% \lstset{language=Java,
%% basicstyle=\ttfamily,
%% keywordstyle=\color{javapurple}\bfseries,
%% stringstyle=\color{javared},
%% commentstyle=\color{javagreen},
%% morecomment=[s][\color{javadocblue}]{/**}{*/},
%% numbers=left,
%% numberstyle=\tiny\color{black},
%% stepnumber=2,
%% numbersep=10pt,
%% tabsize=4,
%% showspaces=false,
%% showstringspaces=false}

%% \begin{lstlisting}

%% \end{lstlisting}


%!TEX root = SpGEMM_Accumulo_HPEC.tex

\section{Performance}
\label{sPerformance}

We evaluate TableMult with two variants of an experiment. 
First we measure the rate of computation as problem size increases.
We define problem size as number of rows in random input graphs 
represented as adjacency tables
and rate of computation as number of partial products processed per second.
Second we repeat the experiment for a fixed size problem with all tables split into two tablets,
allowing Accumulo to scan and write to them in parallel.

%% We evalutate our TableMult implementation with a weak scaling experiment,
%% measuring rate of computation as problem size increases.
%% We define problem size as the number of nodes in random input graphs
%% and rate of computation as the number of partial products processed per second.  
%% We also perform the same tests when input and output tables
%% are split into two tablets each, which allows Accumulo to scan and write to them in parallel.

%Ideally we would include run our tests on more than two tablets to truly test strong scaling, 
%but such tests are infeasible given the laptop's capabilities.

We compare Graphulo TableMult performance to D4M as a baseline because 
a user with one client machine's best alternative is reading input graphs from Accumulo, 
multiplying them at the client, and inserting the result back into Accumulo.

D4M stores tables as Associative Array objects in Matlab.  
Because Assoc Array multplication runs fast in memory, 
D4M bottlenecks on reading data from Accumulo and especially on writing back results.
We consequently expect TableMult to outperform D4M 
because TableMult avoids transferring data out of Accumulo for processing. 

We also expect TableMult to succeed on larger graph sizes than D4M because TableMult
uses a streaming outer product algorithm that does not store input tables in memory.
An alternative D4M implementation would mirror TableMult's streaming outer product algorithm,
enabling D4M to run on larger problem sizes at potentially worse performance.
We therefore imagine the whole-table D4M algorithm as an upper bound on the best performance 
achievable when multiplying Accumulo tables outside Accumulo's infrastructure.

We use the Graph500 unpermuted power law graph generator \cite{bader2006designing} 
to create random input tables,
 %also used in 100M insert/sec paper
such that the first row has high degree (number of columns) 
%The generator creates graphs with a power law structure, such that the first vertex has high degree
and subsequent rows exponentially decrease in degree.
The generator takes SCALE and EdgesPerVertex parameters, creating graphs with 2\textsuperscript{SCALE} 
rows and EdgesPerVertex $\times$ 2\textsuperscript{SCALE} entries.
We fix EdgesPerVertex to 16 and use SCALE to vary problem size. 

%We multiply the transpose of the first table with the second table in our tests.
The following procedure outlines our performance experiment for a given SCALE and either one or two tablets.
\begin{enumerate}
\item Generate two graphs with different random seeds and insert them into Accumulo as adjacency tables via D4M.
\item In the case of two tablets, identify an optimal split point for each input graph
and set the input graphs' table splits equal to that point.
``Optimal'' here means a split point that evenly divides an input graph into two tablets.
\item \label{ePreSplit1} Create an empty output table. For two tablets, pre-split it with 
an optimal input split position recorded from a previous multiplication run.
%% the first input table's split.
%% The split will not be optimal for the output table because the matrix product has a different degree distribution 
%% than that of the input tables, but it is close enough for the purposes of our test.
\item \label{ePreSplit1Compact} Compact the input and output tables 
so that Accumulo redistributes the tables' entries into the assigned tablets.
\item Run and time Graphulo TableMult multiplying the transpose of the first input table with the second.
\item Create, pre-split and compact a new result table for the D4M comparison 
as in step~\ref{ePreSplit1} and~\ref{ePreSplit1Compact}.
\item Run and time the D4M equivalent of TableMult:
 \begin{enumerate}
 \item Scan both input tables into D4M Associative Array objects in Matlab memory.
 \item Convert the string values from Accumulo into numeric values for each Assoc.
 \item Multiply the transpose of the first Assoc with the second Assoc.
 \item Convert the result Assoc back to String values and insert them into Accumulo.
 \end{enumerate}
\end{enumerate}

%\subsection{Environment}
We conduct the experiments on a laptop with 16GB RAM with dual dual-core Intel i7 processors %at 3GHz
running Ubuntu 14.04 linux. We use single-instance Accumulo 1.6.1, Hadoop 2.6.0 and ZooKeeper 3.4.6.
We allocate 2GB of memory to the Accumulo tablet server initially
(allowing growth in 500MB steps),
1GB for native in-memory maps and 256MB for data and index caches.


We chose not to use more than two tablets per table because more threads would run
than the laptop could handle.  Each additional tablet can potentially add the following threads:
\begin{enumerate}
\item Table $\matr{A}^\tr$ server-side scan thread
\item Table $\matr{A}^\tr$ client-side scan thread,

$\quad$ running from RemoteSourceIterator
\item Table $\matr{B}$ server-side scan/multiply thread,

$\quad$ running a TableMult iterator stack
\item Table $\matr{B}$ client-side scan thread, 

$\quad$ running from the initiating client; mostly idle
\item Table $\matr{C}$ server-side write thread
\item Table $\matr{C}$ client-side write thread,

$\quad$ running from RemoteWriteIterator
\item Table $\matr{C}$ server-side minor compaction threads,

$\quad$ running with a Combiner implementing $\oplus$
\end{enumerate}

\begin{figure}[tbh]
%\vspace{-6pt}
\centering
\includegraphics[width=\linewidth]{TableMultRate}
\caption{TableMult Processing Rate vs. Input Table Size}
\label{fTableMultPerf}
\vspace{-4pt}
\end{figure}

We show table $\matr{C}$ sizes and experiment timings in Table~\ref{tResultsParams}
and plot them in Figure~\ref{fTableMultPerf}.
We could not run the D4M comparison past SCALE 15 because $\matr{C}$ does not fit in memory.

For the scaled problem, the best results we could achieve are flat horizontal lines, 
indicating that we maintain the same level of operations per second as problem size increases.
%Graphulo roughly achieves weak scaling, although the two-tablet Graphulo curve shows instability.

One reason we see a downward rate trend at larger problem sizes is that Accumulo
needs to minor compact table $\matr{C}$ in the middle of a TableMult. 
The compactions trigger flushes to disk along with 
the $\oplus$ Combiner that sums partial products written to $\matr{C}$ so far, 
neither of which we include 
in rate measurements. % (in partial products per second).
%% Thus, one explanation for the rate decrease is that 
%% our rate measurements (in partial products per second) do not include summing 
%% and disk flush operations Accumulo performs during minor compaction.

For the fixed size problem, the best results we could achieve are two-tablet rates
double the one-tablet rates for every problem size.
Our experiment shows that Graphulo two-tablet multiply rates perform up to 1.5x better
than one-tablet rates with degraded performance at higher SCALEs.  
We attribute TableMult's shortfall to high processor contention for the four cores as a result of 
the 14 threads that may run concurrently with two tablets on ; in fact,
processor usage hovered near 100\% for all four laptop cores throughout the two-tablet experiments.
We expect better scaling when running our experiment 
in a larger Accumulo cluster that can handle more degrees of parallelism.

%Odd factor: OS frequently killed the Accumulo garbage collector.

\begin{table*}[tb]
%\vspace{-1.75em}
\centering                                                                                                           
\caption{Output Table $\matr{C}$ Sizes and Experiment Timings}% plotted in Figure~\ref{fTableMultPerf}}
\label{tResultsParams}
\begin{threeparttable}[c]
\addtolength{\tabcolsep}{-0.5pt}  
\begin{tabular}{r|ll|ll|ll|ll|ll}
%\setlength{\medmuskip}{1\medmuskip}
\multirow{2}{1.75em}{\adjustbox{angle=30,lap=\width-3.75em}{SCALE}} & \multicolumn{2}{c|}{Entries in Table $\matr{C}$} & \multicolumn{2}{c|}{Graphulo 1 Tablet} & \multicolumn{2}{c|}{D4M 1 Tablet} & \multicolumn{2}{c|}{Graphulo 2 Tablets} & \multicolumn{2}{c}{D4M 2 Tablets} \\
 & PartialProducts\hspace{-0.75em} & AfterSum & Time (s) & Rate (pp/s) & Time (s) & Rate (pp/s) & Time (s) & Rate (pp/s) & Time (s) & Rate (pp/s) \\             
\hline
10 & \num{804989.000} & \num{269404.000} & \num{2.865} & \num{281012.707} & \num{3.018} & \num{266771.720} & \num{2.022} & \num{398174.309} & \num{2.804} & \num{287060.355} \\             
\hline                                                                                                                                                                                                          
11 & \num{2361580.000} & \num{814644.000} & \num{7.758} & \num{304413.622} & \num{8.803} & \num{268259.547} & \num{5.189} & \num{455121.509} & \num{8.718} & \num{270898.575} \\            
\hline                                                                                                                                                                                                          
12 & \num{6816962.000} & \num{2430381.000} & \num{21.984} & \num{310090.248} & \num{26.601} & \num{256270.986} & \num{16.307} & \num{418039.002} & \num{26.182} & \num{260366.279} \\       
\hline                                                                                                                                                                                                          
13 & \num{19111689.000} & \num{7037007.000} & \num{63.969} & \num{298766.256} & \num{150.475} & \num{127009.402} & \num{48.623} & \num{393059.423} & \num{144.156} & \num{132575.978} \\    
\hline                                                                                                                                                                                                          
14 & \num{52656204.000} & \num{20029427.000} & \num{181.506} & \num{290106.916} & \num{579.243} & \num{90905.237} & \num{136.025} & \num{387107.096} & \num{559.271} & \num{94151.551} \\   
\hline                                                                                                                                                                                                          
15 & \num{147104084.000} & \num{58288789.000} & \num{502.864} & \num{292532.774} & \num{2510.389} & \num{58598.135} & \num{393.573} & \num{373765.880} & \num{2559.243} & \num{57479.523} \\
\hline                                                                                                                                                                                                          
16 & \num{400380031.000} & \num{163481262.000} & \num{1390.612} & \num{287916.484} &  &  & \num{1178.111} & \num{339849.273} &  &  \\                                                               
\hline                                                                                                                                                                                                          
17 & \num{1086789275.000} & \num{459198683.000} & \num{4064.990} & \num{267353.526} &  &  & \num{3699.671} & \num{293752.983} &  &  \\                                                              
\hline                                                                                                                                                                                                          
18 & $2.94 \times 10^9$
%\num{2937549526} 
& \num{1280878452.000} & \num{12148.744} & \num{241798.621} &  &  & \num{11369.009} & \num{258382.204} &  &  \\
\end{tabular}
%\begin{tablenotes}
%\tnote{1}
%\item [a] nnz($\matr{C}$) is the number of elements in output table $\matr{C}$ after all partial products sum together.
%\end{tablenotes}
\end{threeparttable}
%\vspace{-2.5em}
\end{table*}

\renewcommand{\times}[0]{\stimes}

%!TEX root =SpGEMM_Accumulo_HPEC.tex

\section{Discussion}
\label{sDiscussion}

\subsection{TableMult Design Alternatives}

A common Accumulo pattern is to run multiple clients that scan disjoint and continuous table sections in parallel.
We avoid this pattern because it increases client software complexity,
whereas we aim to provide a service within Accumulo that works for any client.
Perhaps more importantly, previous work has shown that table scans 
that do not perform significant iterator processing %filtering or other server-side computation 
bottleneck on communication overhead \emph{at the client} related to Apache Thrift serialization \cite{sawyer2013understanding}.
We gain a chance to eliminate this overhead by moving computation to the server,
though we do not currently do so as we use standard Accumulo Scanners and BatchWriters.

%bridge inner and outer product
We also find room to reconsider the inner product SpGEMM formulation from our initial design
because it has an opposite performance profile as Figure~\ref{fInnerOuterSpectrum} depicts: 
inner product bottlenecks on scanning whereas outer product bottlenecks on writing.
At the expense of multiple passes over input matrices, inner product emits entries more efficiently
in that emitted entries are in order and partial products can sum immediately, 
reducing the number of entries written to Accumulo to the minimum possible.
Outer product reads inputs in a single pass
but emits %entries less efficiently in that entries emit 
entries out of order and 
has little chance to pre-sum partial products, 
instead writing individual partial products to the result table.
Table~\ref{tResultsParams} quantifies the additional number of entries outer product writes
for power law inputs as 2.5 to 3 times that of inner product.
In the worst case, multiplying a fully dense $n \times m$ with an $m \times p$ matrix,
outer product would emit $m$ times more entries than inner product.

\begin{figure}[tbh]
\centering
\includegraphics[width=\linewidth]{InnerOuterSpectrum}
\caption{Tradeoffs between Inner and Outer Product}
\label{fInnerOuterSpectrum}
\end{figure}

%LRU cache enables the bridge
Is it possible to blend inner and outer product algorithms and achieve better performance than either method alone?
A definitive answer is future research, though we sketch one possible bridge in Figure~\ref{fInnerOuterSpectrum}.
Suppose we partition the rows of $\matr{A}$ (columns of $\matr{A^\tr}$) into two halves 
and run outer product on each separately.
Each outer product requires one scan over $\matr{B}$ for a total of two passes.
In exchange we increase write locality:
the top half outer product only writes to the top half of $\matr{C}$ and 
the bottom half outer product only writes to the bottom half of $\matr{C}$.
%We could do the same process on table $\matr{B}$ by symmetry.
Selecting the number of partitions along rows of $\matr{A}$ determines
balance between inner product (partitions between every row) and outer product (zero partitions).

A challenge for any hybrid algorithm is mapping it to Accumulo infrastructure.
We chose outer product because it more naturally fits into Accumulo, 
using iterators for one-pass streaming computation, 
BatchWriters to handle unsorted entry emission and Combiners to defer summation.
%Hybrid solutions might consider locality groups or transpose tables to enable column-oriented scanning
%and the distribution of tablets to tablet server cost modeling to 
We may realize greater performance by considering data placement among tablet servers, 
but such a consideration would require accessing and perhaps manipulating
Accumulo's internal state and non-public API.
We suggest this paper's approach as a balance between top performance and implementation stability.


\subsection{TableMult in Algorithms}
Several optimization opportunities exist for TableMult as a primitive in larger algorithms.
Suppose we have a program $\matr{E} = \matr{AB}; \matr{F} = \matr{CD}; \matr{G} = \matr{EF}$.
We would save a round trip to disk if we could mark tables $\matr{E}$ and $\matr{F}$ as 
``temporary tables,'' i.e. tables intermediate to an algorithm that should be held in memory 
and not written to Hadoop if possible.

A \emph{pipelining} optimization streams entries from a TableMult to downstream computations. 
Outer product pipelining is difficult
because it cannot guarantee all partial products for any particular element 
are written to table $\matr{C}$ until it finishes.
Inner product's write locality makes it easier to pipeline.
More ambitiously, a \emph{loop fusion} optimization merges iterator stacks 
for two computations into one. 

Optimizing computation on NoSQL databases is challenging in the general case because
NoSQL databases mostly exclude query planner features 
customary of SQL databases in exchange for raw performance.
NewSQL databases aim in part to achieve the best of both worlds---performance and query planning \cite{grolinger2013data}.
We aspire to make a small step for Accumulo in the direction of NewSQL with current Graphulo research.



\subsection{Related Work} %Analogy to MapReduce with Accumulo Scanners, Iterators and BatchWriters:
%\todo[inline]{Cannon's algorithm, other SpGEMM}
Bulu\c{c} and Gilbert studied message passing algorithms for SpGEMM
such as Sparse SUMMA, most of which use 2D block decompositions \cite{buluc2012parallel}.
Unfortunately, 2D decompositions are difficult in Accumulo 
and message passing even more so.
In this work we use Accumulo's native 1D decomposition along rows 
and no tablet server communication
other than shuffling partial products to tablets of $\matr{C}$ via BatchWriters.


Our outer product method could have been implemented in MapReduce %x\cite{dean2008mapreduce} 
on Hadoop or its successor YARN \cite{vavilapalli2013apache}.
In fact, there is a natural analogy from how we process data in Accumulo to MapReduce:
the map phase scans rows from $\matr{A^\tr}$ and $\matr{B}$
and generates a list of partial products from TwoTableIterator;
the shuffle phase sends partial products to the correct tablet of $\matr{C}$ via BatchWriters;
the reduce phase sums partial products using Accumulo Combiners.
Examining the conditions on which MapReduce outperforms an Accumulo-only solution
is worthy future work.




\section{Conclusions}
\label{sConclusions}

In this work we showcase the design of TableMult, a Graphulo implementation of the 
SpGEMM GraphBLAS linear algebra kernal server-side on Accumulo tables.
We compare inner and outer approaches for SpGEMM and show how outer product 
better suits the Accumulo iterator environment.
Performance experiments show good weak scaling and hint at strong scaling,
although repeating our experiments on a larger cluster is necessary to confirm.

Current research is to implement the remaining GraphBLAS kernels 
and develop algorithms calling them, % atop the Graphulo library,
ultimately delivering a Graphulo linear algebra library 
as a pattern for server-side computation
to the Accumulo community.


%\begin{figure}[!t]
%\centering
%\includegraphics[width=2.5in]{myfigure}
% where an .eps filename suffix will be assumed under latex,
% and a .pdf suffix will be assumed for pdflatex; or what has been declared
% via \DeclareGraphicsExtensions.
%\caption{Simulation Results}
%\label{fig_sim}
%\end{figure}

% Note that IEEE typically puts floats only at the top, even when this
% results in a large percentage of a column being occupied by floats.


% An example of a double column floating figure using two subfigures.
% (The subfig.sty package must be loaded for this to work.)
% The subfigure \label commands are set within each subfloat command, the
% \label for the overall figure must come after \caption.
% \hfil must be used as a separator to get equal spacing.
% The subfigure.sty package works much the same way, except \subfigure is
% used instead of \subfloat.
%
%\begin{figure*}[!t]
%\centerline{\subfloat[Case I]\includegraphics[width=2.5in]{subfigcase1}%
%\label{fig_first_case}}
%\hfil
%\subfloat[Case II]{\includegraphics[width=2.5in]{subfigcase2}%
%\label{fig_second_case}}}
%\caption{Simulation results}
%\label{fig_sim}
%\end{figure*}
%
% Note that often IEEE papers with subfigures do not employ subfigure
% captions (using the optional argument to \subfloat), but instead will
% reference/describe all of them (a), (b), etc., within the main caption.


% An example of a floating table. Note that, for IEEE style tables, the
% \caption command should come BEFORE the table. Table text will default to
% \footnotesize as IEEE normally uses this smaller font for tables.
% The \label must come after \caption as always.
%
%\begin{table}[!t]
%% increase table row spacing, adjust to taste
%\renewcommand{\arraystretch}{1.3}
% if using array.sty, it might be a good idea to tweak the value of
% \extrarowheight as needed to properly center the text within the cells
%\caption{An Example of a Table}
%\label{table_example}
%\centering
%% Some packages, such as MDW tools, offer better commands for making tables
%% than the plain LaTeX2e tabular which is used here.
%\begin{tabular}{|c||c|}
%\hline
%One & Two\\
%\hline
%Three & Four\\
%\hline
%\end{tabular}
%\end{table}


% Note that IEEE does not put floats in the very first column - or typically
% anywhere on the first page for that matter. Also, in-text middle ("here")
% positioning is not used. Most IEEE journals/conferences use top floats
% exclusively. Note that, LaTeX2e, unlike IEEE journals/conferences, places
% footnotes above bottom floats. This can be corrected via the \fnbelowfloat
% command of the stfloats package.

% conference papers do not normally have an appendix


% use section* for acknowledgement
%% \section*{Acknowledgment}

%% The authors wish to thank the entire Graphulo team at MIT CSAIL and
%% MIT Lincoln Laboratory. We also thank the GraphBLAS contributors and
%% National Science Foundation for their generous ongoing support of this program.

% trigger a \newpage just before the given reference
% number - used to balance the columns on the last page
% adjust value as needed - may need to be readjusted if
% the document is modified later
%\IEEEtriggeratref{8}
% The "triggered" command can be changed if desired:
%\IEEEtriggercmd{\enlargethispage{-5in}}

% references section

% can use a bibliography generated by BibTeX as a .bbl file
% BibTeX documentation can be easily obtained at:
% http://www.ctan.org/tex-archive/biblio/bibtex/contrib/doc/
% The IEEEtran BibTeX style support page is at:
% http://www.michaelshell.org/tex/ieeetran/bibtex/
%\bibliographystyle{IEEEtran}
% argument is your BibTeX string definitions and bibliography database(s)
%\bibliography{IEEEabrv,../bib/paper}
%
% <OR> manually copy in the resultant .bbl file
% set second argument of \begin to the number of references
% (used to reserve space for the reference number labels box)


%\begin{thebibliography}{1}
%
%\bibitem{NIST}
%P.~Mell and T.~Grace, \emph{The NIST Definition of Cloud Computing},
%NIST Special Publication 800-145
%
%\end{thebibliography}

\bibliographystyle{IEEEtran}

%\titlespacing*{\section}{0pt}{-80pt}{40pt}
\bibliography{10_bibliography}

\balance
%\appendix
%\section*{Performance Numbers}


\end{document}


